\thispagestyle{fancy}
\paginiciales{RESUMEN}
\addcontentsline{toc}{chapter}{RESUMEN}
\vspace{5mm}

Esta tesis describe, crea y valida el modelamiento cinemático y dinámico de un robot paralelo tipo delta de 3 grados de libertad. 

En los últimos años ha ido ganando popularidad el uso de la cinemática junto con la dinámica para el control de robots paralelos. El uso de robots paralelos dedicados a las operaciones de ‘pick and place’ abrió una serie de nuevas perspectivas en el dominio del manejo rápido y preciso de objetos ligeros. Estos son utilizados en diversas áreas en la industria, tales como la agronomía, fabricación, laboratorios farmacéuticos, etc.

Las problemáticas que se identifican acerca del robot delta y que se abordan en esta tesis son principalmente: los sistemas robóticos gestionan una gran complejidad, distinta para cada tarea a realizar y esto trae como consecuencia lentitud y grandes costos en el desarrollo  e implementación de la robótica a escala mundial; los esquemas de controles de robots basados solo en la cinemática de posición no son suficientes para una excelente precisión y  la dificultad de establecer un modelo dinámico simple que pueda calcularse fácilmente en tiempo real. Por ende, en este trabajo se propone el uso un middleware gratuito orientado a la reutilización de código y control de robots; dos métodos para la modelación cinemática y dinámica; espacio de trabajo donde ejecuta las tareas el robot; visualización en 3D de las partes mecánicas; simulación de trayectorias para comprobar los métodos y validación de los modelos por medio de software de simulación educacional.  

Los resultados muestran que el modelamiento cinemático y dinámico de los dos métodos dan valores idénticos y son validos según el software de simulación. Las diferencias despreciables entre los resultados de los modelos y el software de simulación se deben a que este último solo se aproxima al modelo, en otras palabras, existen simplificaciones en las juntas que provocan pequeñas diferencias en los resultados teóricos.

\vfill
\noindent\textbf{Palabras clave:} Robot Delta, Robot Operating System, ROS, ADAMS, Robot Paralelo, Cinemática, Dinámica, Jacobiano, RViz, Espacio de Trabajo, Robótica.
